%----------------------------------------------------------------------------------------
%    PACKAGES AND THEMES
%----------------------------------------------------------------------------------------

\documentclass[aspectratio=169,xcolor=dvipsnames]{beamer}
\usetheme{SimplePlus}

\usepackage{hyperref}
\usepackage{graphicx} % Allows including images
\usepackage{booktabs} % Allows the use of \toprule, \midrule and \bottomrule in tables

%----------------------------------------------------------------------------------------
%    TITLE PAGE
%----------------------------------------------------------------------------------------

\title{Stochastic Optimal Control Matching}
\subtitle{Carles Domingo-Enrich, Jiequn Han, Brandon Amos, Joan Bruna, Ricky T. Q. Chen}

\author{Thomas Mousseau}

% \institute
% {
%     Department of Computer Science and Information Engineering \\
%     National Taiwan University % Your institution for the title page
% }
\date{\today} % Date, can be changed to a custom date

%----------------------------------------------------------------------------------------
%    PRESENTATION SLIDES
%----------------------------------------------------------------------------------------

\begin{document}

\begin{frame}
    % Print the title page as the first slide
    \titlepage
\end{frame}

\begin{frame}{Overview}
    % Throughout your presentation, if you choose to use \section{} and \subsection{} commands, these will automatically be printed on this slide as an overview of your presentation
    \tableofcontents
\end{frame}

%------------------------------------------------
\section{Setup and Preliminaries}

\begin{frame}{Evolution of Generative Models}
    \begin{center}
        \begin{minipage}{0.9\textwidth}
            \vspace{0.3cm}
            
            % Compact timeline with years and descriptions
            \small
            \begin{tabular}{@{}l@{\hspace{0.8cm}}p{0.75\textwidth}@{}}
                \textbf{2020} & \textbf{DDPM:} Denoising Diffusion Probabilistic Models interpret generation as reversing a discrete noise-adding process, learning to denoise at each step. They produced high-quality samples but required thousands of slow sampling steps. \\[0.4cm]
                
                \textbf{2021} & \textbf{Score-based Models:} Score-based generative models extended diffusion to continuous-time SDEs, learning the score function ($\nabla_x \log p_t(x)$) to reverse a stochastic diffusion process. This unified diffusion with stochastic control, allowed probability flow ODEs, and sped up sampling. \\[0.4cm]
                
                \textbf{2023} & \textbf{Flow Matching:} Flow matching views generation as learning a deterministic ODE vector field that directly transports a simple distribution (e.g., Gaussian) to data. This removed stochasticity and significantly improved efficiency compared to diffusion/score methods. \\[0.4cm]
                % \textbf{2023} & \textbf{SOCM:} Stochastic Optimal Control Matching reframes this transport problem as an optimal control task, where the learned control steers the process from a simple distribution to data. Its key contribution is a variance-reduced least-squares objective, yielding more stable and accurate training than prior approaches.
            \end{tabular}
            
            \vspace{0.3cm}
        \end{minipage}
    \end{center}
\end{frame}



\begin{frame}{What is a Stochastic Control Problem?}
    A stochastic control problem involves finding an optimal control policy to steer a dynamical system under uncertainty.
    
    \vspace{0.3cm}
    
    \begin{block}{Key Components}
        \begin{itemize}
            \item \textbf{State Process:} $X_t \in \mathbb{R}^d$ (position in state space at time $t$)
            \item \textbf{Control Process:} $u_t \in \mathbb{R}^d$ (action/decision at time $t$)
            \item \textbf{Noise Process:} $W_t$ (random disturbances, typically Brownian motion)
        \end{itemize}
    \end{block}
    
    \begin{columns}[t]
        \column{0.45\textwidth}
        \begin{alertblock}{Dynamics (SDE)}
            \vspace{-0.1cm}
            \begin{equation}
            dX_t = f(X_t, u_t, t) dt + g(X_t, t) dW_t
            \end{equation}
            \vspace{-0.2cm}
        \end{alertblock}
        
        \column{0.52\textwidth}
        \begin{alertblock}{Cost Function}
            \vspace{-0.1cm}
            \begin{equation}
            J(u) = \mathbb{E}\left[\int_0^T L(X_t, u_t, t) dt + \Phi(X_T)\right]
            \end{equation}
            \vspace{-0.2cm}
        \end{alertblock}
    \end{columns}
\end{frame}

\begin{frame}{The Goal: Finding Optimal Control}
    \begin{block}{Optimal Control $u^*$}
        Find the control policy $u^*$ that minimizes the expected cost: $u^* = \arg\min_u J(u)$
    \end{block}
    
    \vspace{0.3cm}
    
    \begin{block}{Classical Approaches}
        \begin{itemize}
            \item \textbf{Hamilton-Jacobi-Bellman (HJB) equation:} Partial differential equation approach
            \item \textbf{Pontryagin's Maximum Principle:} Necessary conditions for optimality
            \item \textbf{Dynamic Programming:} Discrete-time recursive approach
        \end{itemize}
    \end{block}
    
    \vspace{0.3cm}
    
    \begin{alertblock}{Challenge}
        These classical methods become computationally intractable in high dimensions due to the \textit{curse of dimensionality}.
    \end{alertblock}
\end{frame}

%------------------------------------------------

%------------------------------------------------
% \section{Setup and preliminaries}
%------------------------------------------------

%------------------------------------------------
\section{Stochastic Optimal Control Matching}


\begin{frame}{Introducing Stochastic Optimal Control Matching}
    SOCM offers a more principled, stable, and accurate way to learn generative dynamics by blending stochastic control theory with modern matching-based generative modeling.
    
    \vspace{0.4cm}
    
    \begin{block}{Key Novel Contributions}
        \begin{enumerate}
            \item \textbf{Controlled Stochastic Process:} Views the generation process as a controlled stochastic process bridging a simple distribution to data.
            
            \vspace{0.2cm}
            
            \item \textbf{Least-Squares Matching:} Learning the control via least-squares matching—a stable, convex regression objective.
            
            \vspace{0.2cm}
            
            \item \textbf{Joint Optimization:} Optimizing control and variance-reducing reparameterization matrices simultaneously, for efficient learning.
            
            \vspace{0.2cm}
            
            \item \textbf{Path-wise Reparameterization:} Introducing a path-wise reparameterization trick, boosting gradient estimation quality.
        \end{enumerate}
    \end{block}
\end{frame}

\begin{frame}{The SOCM Framework}
    \begin{block}{Key Components}
        \begin{itemize}
            \item \textbf{Controlled Stochastic Process:} $dX_t = f(X_t, u_t, t) dt + g(X_t, t) dW_t$
            \item \textbf{Cost Function:} $J(u) = \mathbb{E}\left[\int_0^T L(X_t, u_t, t) dt + \Phi(X_T)\right]$
            \item \textbf{Control Policy:} $u^* = \arg\min_u J(u)$
        \end{itemize}
    \end{block}
    
    \vspace{0.3cm}
    
    \begin{block}{Learning Objective}
        Minimize the expected cost using a least-squares regression objective:
        \begin{equation}
        L(u) = \mathbb{E}\left[\|X_T - X_{data}\|^2\right]
        \end{equation}
    \end{block}
\end{frame}

\begin{frame}{Advantages of Controlled Stochastic Processes}
    Why model generation as a controlled stochastic process rather than a fixed deterministic flow?
    
    \vspace{0.4cm}
    
    \begin{block}{Key Advantages}
        \begin{enumerate}
            \item \textbf{Powerful Theoretical Tools:} Access to established methods from stochastic optimal control theory like Hamilton-Jacobi-Bellman equations and dynamic programming.
            
            \vspace{0.2cm}
            
            \item \textbf{Learnable Trajectories:} The trajectory of the distribution is no longer fixed (unlike hand-designed noise schedulers in DDPM) and can be learned and optimized to improve efficiency and quality.
            
            \vspace{0.2cm}
            
            \item \textbf{Enhanced Interpretability:} The objective becomes finding an optimal control policy, which provides clear insight into how the model moves from its initial distribution to the final complex data distribution.
        \end{enumerate}
    \end{block}
    
    \vspace{0.3cm}
    
    \begin{alertblock}{Result}
        These advantages enable SOCM to achieve better sample quality and training stability compared to traditional approaches.
    \end{alertblock}
\end{frame}

%------------------------------------------------

%------------------------------------------------
\section{Experiments and results}
%------------------------------------------------

%------------------------------------------------
\section{Conclusion}
%------------------------------------------------


% \begin{frame}{Bullet Points}
%     \begin{itemize}
%         \item Lorem ipsum dolor sit amet, consectetur adipiscing elit
%         \item Aliquam blandit faucibus nisi, sit amet dapibus enim tempus eu
%         \item Nulla commodo, erat quis gravida posuere, elit lacus lobortis est, quis porttitor odio mauris at libero
%         \item Nam cursus est eget velit posuere pellentesque
%         \item Vestibulum faucibus velit a augue condimentum quis convallis nulla gravida
%     \end{itemize}
% \end{frame}

%------------------------------------------------

\begin{frame}{Blocks of Highlighted Text}
    In this slide, some important text will be \alert{highlighted} because it's important. Please, don't abuse it.

    \begin{block}{Block}
        Sample text
    \end{block}

    \begin{alertblock}{Alertblock}
        Sample text in red box
    \end{alertblock}

    \begin{examples}
        Sample text in green box. The title of the block is ``Examples".
    \end{examples}
\end{frame}

%------------------------------------------------

\begin{frame}{Multiple Columns}
    \begin{columns}[c] % The "c" option specifies centered vertical alignment while the "t" option is used for top vertical alignment

        \column{.45\textwidth} % Left column and width
        \textbf{Heading}
        \begin{enumerate}
            \item Statement
            \item Explanation
            \item Example
        \end{enumerate}

        \column{.45\textwidth} % Right column and width
        Lorem ipsum dolor sit amet, consectetur adipiscing elit. Integer lectus nisl, ultricies in feugiat rutrum, porttitor sit amet augue. Aliquam ut tortor mauris. Sed volutpat ante purus, quis accumsan dolor.

    \end{columns}
\end{frame}

%------------------------------------------------

\begin{frame}{Table}
    \begin{table}
        \begin{tabular}{l l l}
            \toprule
            \textbf{Treatments} & \textbf{Response 1} & \textbf{Response 2} \\
            \midrule
            Treatment 1         & 0.0003262           & 0.562               \\
            Treatment 2         & 0.0015681           & 0.910               \\
            Treatment 3         & 0.0009271           & 0.296               \\
            \bottomrule
        \end{tabular}
        \caption{Table caption}
    \end{table}
\end{frame}

%------------------------------------------------

\begin{frame}{Theorem}
    \begin{theorem}[Mass--energy equivalence]
        $E = mc^2$
    \end{theorem}
\end{frame}

%------------------------------------------------

\begin{frame}{Figure}
    Uncomment the code on this slide to include your own image from the same directory as the template .TeX file.
    %\begin{figure}
    %\includegraphics[width=0.8\linewidth]{test}
    %\end{figure}
\end{frame}

%------------------------------------------------

\begin{frame}[fragile] % Need to use the fragile option when verbatim is used in the slide
    \frametitle{Citation}
    An example of the \verb|\cite| command to cite within the presentation:\\~

    This statement requires citation \cite{p1}.
\end{frame}

%------------------------------------------------

\begin{frame}{References}
    \footnotesize
    \bibliography{reference.bib}
    \bibliographystyle{apalike}
\end{frame}

%------------------------------------------------

\begin{frame}
    \Huge{\centerline{\textbf{The End}}}
\end{frame}

%----------------------------------------------------------------------------------------

\end{document}