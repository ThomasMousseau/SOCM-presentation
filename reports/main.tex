%----------------------------------------------------------------------------------------
%    PACKAGES AND THEMES
%----------------------------------------------------------------------------------------

\documentclass[aspectratio=169,xcolor=dvipsnames]{beamer}
\setbeameroption{show notes} %TODO: Thomas a enlever avant la presentation
\usetheme{SimplePlus}

\usepackage{hyperref}
\usepackage{graphicx} % Allows including images
\usepackage{booktabs} % Allows the use of \toprule, \midrule and \bottomrule in tables
\usepackage{array} % Allows >{\centering\arraybackslash} in tabular

%----------------------------------------------------------------------------------------
%    TITLE PAGE
%----------------------------------------------------------------------------------------

\title{Stochastic Optimal Control Matching}
\subtitle{Carles Domingo-Enrich, Jiequn Han, Brandon Amos, Joan Bruna, Ricky T. Q. Chen}

\author{Thomas Mousseau}

% \institute
% {
%     Department of Computer Science and Information Engineering \\
%     National Taiwan University % Your institution for the title page
% }
\date{\today} % Date, can be changed to a custom date

%----------------------------------------------------------------------------------------
%    PRESENTATION SLIDES
%----------------------------------------------------------------------------------------

\begin{document}

\begin{frame}
    % Print the title page as the first slide
    \titlepage
\end{frame}

\begin{frame}{Overview}
    % Throughout your presentation, if you choose to use \section{} and \subsection{} commands, these will automatically be printed on this slide as an overview of your presentation
    \tableofcontents
\end{frame}

%------------------------------------------------
\section{Setup and Preliminaries}

%TODO: J'aimerais beaucoup glisser les Continuous Normalizing Flows a.k.a Neural ODE avant DDPM et le reste puisque c'est vrm un paper essential a research field
\begin{frame}{Neural ODE: Continuous Normalizing Flows}

    \begin{block}{Continuous Normalizing Flows (CNF)}
        CNFs model complex distributions by transforming a simple distribution (e.g., Gaussian) through a continuous-time ODE. The transformation is defined by a neural network that learns the dynamics of the flow.
    \end{block}
    
    \vspace{0.3cm}
    
    \begin{alertblock}{Key Idea}
        Instead of discrete steps, CNFs use a continuous-time approach to model the evolution of the distribution, allowing for more flexible and expressive transformations.
    \end{alertblock}

    \note{Cette slide n'est pas complete, je voudrais ici presenter le debut de ce milieu soit le Neural ODEs, le but qu'ils comblent dans le milieu et leur premiere implementation soit CNFs.
            De plus, je veux aussi expliquer les Adjoint Methods ici puisque elles sont recurrente dans tous le reste du SOCM papier}
\end{frame}

\begin{frame}{Evolution of Generative Models}
    \begin{center}
        \begin{minipage}{0.9\textwidth}
            \vspace{0.3cm}
            
            \small
            \begin{tabular}{@{}l@{\hspace{0.8cm}}p{0.75\textwidth}@{}}
                \textbf{2020} & \textbf{DDPM:} Denoising Diffusion Probabilistic Models interpret generation as reversing a discrete noise-adding process, learning to denoise at each step. They produced high-quality samples but required thousands of slow sampling steps. \\[0.4cm]
                
                \textbf{2021} & \textbf{Score-based Models:} Score-based generative models extended diffusion to continuous-time SDEs, learning the score function ($\nabla_x \log p_t(x)$) to reverse a stochastic diffusion process. This unified diffusion with stochastic control, allowed probability flow ODEs, and sped up sampling. \\[0.4cm]
                
                \textbf{2023} & \textbf{Flow Matching:} Flow matching views generation as learning a deterministic ODE vector field that directly transports a simple distribution (e.g., Gaussian) to data. This removed stochasticity and significantly improved efficiency compared to diffusion/score methods. \\[0.4cm]
            \end{tabular}
            
            \vspace{0.3cm}
        \end{minipage}
    \end{center}
\end{frame}

\begin{frame}{What is a Stochastic Control Problem?}
    \vspace{-0.7cm}
    \begin{columns}[t]
        \column{0.48\textwidth}
        \begin{alertblock}{Dynamics (SDE)}
            \begin{equation}
            dx_t = \underbrace{u_t dt}_{\text{drift coefficient}} + \underbrace{dw_t}_{\text{diffusion coefficient}}
            \end{equation}
        \end{alertblock}
        
    \vspace{0.1cm}
        
        \column{0.49\textwidth}
        \begin{alertblock}{Cost Function}
            \vspace{-0.1cm}
            \begin{equation}
            J(u) = \mathbb{E}\left[\int_0^T L(x_t, u_t, t) dt + \Phi(x_T)\right]
            \end{equation}
            \vspace{-0.2cm}
        \end{alertblock}
    \end{columns}
    
    \begin{block}{Key Components}
        \small
        \begin{itemize}
            \item \textbf{State Process:} $x_t \in \mathbb{R}^d$ (position in state space at time $t$)
            \item \textbf{Control Process:} $u_t \in \mathbb{R}^d$ (action/decision at time $t$)
            \item \textbf{Noise Process:} $w_t$ (random disturbances, typically Brownian motion)
            \item \textbf{Running Cost:} $L(x_t, u_t, t)$ (cost accumulated over time)
            \item \textbf{Terminal Cost:} $\Phi(x_T)$ (cost at the final time $T$)
        \end{itemize}
    \end{block}

    \note{
    \textcolor{red}{A stochastic control problem involves finding an optimal control policy to steer a dynamical system under uncertainty while minimizing the expected cost.}
    \\
    \textcolor{orange}{A voir si je rajouter une autre slide qui presente la solution avec HJB + Feynman-Kac approache.}
    }
\end{frame}

\begin{frame}{The Goal: Finding Optimal Control}
    \begin{block}{Optimal Control $u^*$}
        Find the control policy $u^*$ that minimizes the expected cost: $u^* = \arg\min_u J(u)$
    \end{block}
    
    \vspace{0.3cm}
    
    \begin{block}{Classical Approaches}
        \begin{itemize}
            \item \textbf{Hamilton-Jacobi-Bellman (HJB) equation:} Partial differential equation approach
            \item \textbf{Pontryagin's Maximum Principle:} Necessary conditions for optimality
            \item \textbf{Dynamic Programming:} Discrete-time recursive approach
        \end{itemize}
    \end{block}
    
    \vspace{0.3cm}
    
    \begin{alertblock}{Challenge}
        These classical methods become computationally intractable in high dimensions due to the \textit{curse of dimensionality}.
    \end{alertblock}
\end{frame}



%TODO: A voir si je veux expliquer ce qu'est une Adjoint Method?!?!??!?!?!?

%TODO: A voir encore si je ne veux pas faire un lien avec RL ou meme directement presente la HJB et demontrer que ls Jacobienne devient difficulement calculate (prix computationnel trop eleve) a plusieurs dimensions
%! For continuous-time problems with low-dimensional state spaces, the standard approach to learn
%! the optimal control is to solve the Hamilton-Jacobi-Bellman (HJB) partial differential equation
%! (PDE) by gridding the space and using classical numerical methods. For high-dimensional problems,
%! a large number of works parameterize the control using a neural network and train it applying a
%! stochastic optimization algorithm on a loss function. These methods are known as Iterative Diffusion
%! Optimization (IDO) techniques [59] (see Subsec. 2.2).


\begin{frame}{Classical vs Neural Solutions}
    \vspace{-0.5cm}
    \begin{columns}[t]
        \column{0.60\textwidth}
        \begin{alertblock}{Classical: HJB PDE}
            \vspace{-0.1cm}
            \small
            \textbf{Hamilton-Jacobi-Bellman Equation:}
            \begin{equation}
            \frac{\partial V}{\partial t} + \min_u \left[ L(x,u,t) + \frac{\partial V}{\partial x} f(x,u,t) + \frac{1}{2}\text{tr}(\sigma^T \nabla^2 V \sigma) \right] = 0
            \end{equation}
            
            \vspace{0.2cm}
            \textbf{Solution Process:}
            \begin{itemize}
                \item Discretize state space on grid
                \item Use finite difference methods
                \item Solve backward in time from $V(x,T) = \Phi(x)$
            \end{itemize}
        \end{alertblock}
        
        \column{0.42\textwidth}
        \begin{block}{Modern: Neural PDE Solvers}
            \vspace{-0.1cm}
            \small
            \textbf{Neural Network Approximation:}
            \begin{equation}
            V(x,t) \approx V_\theta(x,t)
            \end{equation}
            
            \vspace{0.2cm}
            \textbf{Solution Process:}
            \begin{itemize}
                \item Parameterize value function with NN
                \item \textcolor{blue}{Physics-informed loss}: HJB residual
                \item \textcolor{blue}{Automatic differentiation} for gradients
                \item Train end-to-end with SGD
            \end{itemize}
        \end{block}
    \end{columns}
    
    \vspace{0.3cm}
    
    \begin{block}{Key Differences}
        \begin{itemize}
            \item \textbf{Scalability:} Classical methods suffer from \textcolor{red}{curse of dimensionality} (grid size $\propto d^n$), while neural methods scale \textcolor{ForestGreen}{naturally to high dimensions}
            
            \item \textbf{Accuracy:} Classical methods provide \textcolor{blue}{theoretical convergence guarantees}, neural methods offer \textcolor{orange}{empirical flexibility} but may lack guarantees
            
            \item \textbf{Computation:} Classical requires \textcolor{red}{expensive grid computation}, neural requires \textcolor{ForestGreen}{gradient-based optimization}
        \end{itemize}
    \end{block}

    \note{
    \textcolor{blue}{\textbf{HJB Equation:}} The Hamilton-Jacobi-Bellman equation provides the theoretical foundation for solving stochastic optimal control problems by characterizing the value function $V(x,t)$ as the solution to a nonlinear PDE. \\
    \vspace{0.1cm}
    \textcolor{ForestGreen}{\textbf{Neural PDE Solvers:}} Modern approaches use neural networks to approximate the value function directly, avoiding discretization and enabling high-dimensional problems. The network is trained to satisfy the HJB equation through physics-informed loss functions.
    }
\end{frame}

%------------------------------------------------
\section{Stochastic Optimal Control Matching}

\begin{frame}{Reasons behind SOCM (1/2)}
    Many fundamental tasks in machine learning can be naturally cast as stochastic optimal control problems, highlighting the importance of efficient SOC methods.
    
    \vspace{0.4cm}
    
    \begin{block}{Key ML Applications of SOC}
        \begin{itemize}
            \item \textbf{Reward fine-tuning of diffusion and flow models:} Optimizing generation quality using reward signals
            
            \vspace{0.2cm}
            
            \item \textbf{Conditional sampling on diffusion and flow models:} Steering generation towards specific conditions or constraints
            
            \vspace{0.2cm}
            
            \item \textbf{Sampling from unnormalized densities:} Efficiently drawing samples from complex, intractable distributions
            
            \vspace{0.2cm}
            
            \item \textbf{Importance sampling of rare events in SDEs:} Computing probabilities of low-probability but critical events
        \end{itemize}
    \end{block}
    
    \vspace{0.3cm}
    
    \begin{alertblock}{Key Insight}
        The prevalence of SOC formulations in modern ML motivates the need for more efficient and stable solving methods.
    \end{alertblock}
\end{frame}

\begin{frame}{Reasons behind SOCM (2/2)}
    Current SOC methods suffer from optimization challenges that limit their effectiveness.
    
    \vspace{0.3cm}
    
    \begin{columns}[t]
        \column{0.48\textwidth}
        \begin{alertblock}{Current SOC Methods}
            \begin{itemize}
                \item Use \textbf{adjoint methods} (like CNFs)
                \item Yield \textbf{non-convex} function landscapes
                \item Difficult optimization with local minima
                \item Unstable training dynamics
            \end{itemize}
        \end{alertblock}
        
        \column{0.48\textwidth}
        \begin{block}{Diffusion Models Success}
            \begin{itemize}
                \item Use \textbf{least-squares loss}
                \item Create \textbf{convex} functional landscapes
                \item Stable and reliable optimization
                \item Excellent empirical performance
            \end{itemize}
        \end{block}
    \end{columns}
    
    \vspace{0.4cm}
    
    \begin{block}{SOCM's Innovation}
        \textbf{Goal:} Develop least-squares loss formulations for SOC problems, combining the expressiveness of stochastic control with the optimization stability of diffusion models.
    \end{block}
\end{frame}


\begin{frame}{SOCM in Context: Optimization Landscapes}
    \vspace{0.5cm}
    \begin{table}
        \centering
        \begin{tabular}{>{\centering\arraybackslash}p{0.25\textwidth}>{\centering\arraybackslash}p{0.35\textwidth}>{\centering\arraybackslash}p{0.35\textwidth}}
            \toprule
            \textbf{Task} & \textbf{Non-convex} & \textbf{Least Squares} \\
            \midrule
            Generative Modeling & Maximum Likelihood CNFs & Diffusion models and Flow Matching \\
            Stochastic Optimal Control & Adjoint Methods & \textcolor{red}{\textbf{Stochastic Optimal Control Matching}} \\
            \bottomrule
        \end{tabular}
    \end{table}
\end{frame}

\begin{frame}{Introducing Stochastic Optimal Control Matching}
    SOCM offers a more principled, stable, and accurate way to learn generative dynamics by blending stochastic control theory with modern matching-based generative modeling.
    
    \vspace{0.4cm}
    
    \begin{block}{Key Novel Contributions}
        \begin{enumerate}
            \item \textbf{Controlled Stochastic Process:} Views the generation process as a controlled stochastic process bridging a simple distribution to data.
            
            \vspace{0.2cm}
            
            \item \textbf{Least-Squares Matching:} Learning the control via least-squares matching, a stable and convex regression objective.
            
            \vspace{0.2cm}
            
            \item \textbf{Joint Optimization:} Optimizing control and variance-reducing reparameterization matrices simultaneously, for efficient learning.
            
            \vspace{0.2cm}
            
            \item \textbf{Path-wise Reparameterization:} Introducing a path-wise reparameterization trick, boosting gradient estimation quality.
        \end{enumerate}
    \end{block}
\end{frame}


\begin{frame}{The SOCM Framework}
    \small
    \begin{block}{SOCM Loss Function}
        The Stochastic Optimal Control Matching objective is defined as:
        \begin{equation}
        \mathcal{L}_{SOCM}(u, M) := \mathbb{E}\left[\frac{1}{T}\int_0^T \|u(X^v_t, t) - w(t, v, X^v, B, M_t)\|^2 dt \times \alpha(v, X^v, B)\right]
        \end{equation}
    \end{block}

    \begin{block}{Where:}
        \begin{itemize}
            \item $X^v$ is the process controlled by $v$: 
            \begin{equation}
            dX^v_t = (b(X^v_t, t) + \sigma(t)v(X^v_t, t)) dt + \sqrt{\lambda}\sigma(t) dB_t \text{, with } X^v_0 \sim p_0
            \end{equation} 
            \item $u(X^v_t, t)$ is the control policy being learned
            \item $w(t, v, X^v, B, M_t)$ is the target matching function
            \item $\alpha(v, X^v, B)$ is a weighting function
        \end{itemize}
    \end{block}
    \note{Ici, je dois presenter en details la matrice Mt, B ainsi que w. De plus, je veux expliquer l'intuition du path-wise reparameterization trick, a novel
            technique to obtain low-variance estimates of the gradient of the conditional expectation of a functional of
            a random process with respect to its initial value}
\end{frame}

\begin{frame}{SOCM Algorithm}
    \begin{figure}
        \centering
        \includegraphics[width=0.95\textwidth]{figures/SOCM_algo.png}
        \caption{Stochastic Optimal Control Matching (SOCM) Algorithm}
    \end{figure}
\end{frame}

%------------------------------------------------

%------------------------------------------------

\section{Experiments and results}
\begin{frame}{Experimental Results (1/2)}
    \begin{figure}
        \centering
        \includegraphics[width=0.95\textwidth]{figures/plots_1.png}
    \end{figure}
\end{frame}

\begin{frame}{Experimental Results (2/2)}
    \begin{figure}
        \centering
        \includegraphics[width=0.95\textwidth]{figures/plots_2.png}
    \end{figure}
    \note{At the end of training, SOCM obtains the lowest L2
        error, improving over all existing methods by a factor of
        around ten. The two SOCM ablations come in second and third by a substantial difference, which underlines
        the importance of the path-wise reparameterization trick.
        \\
        \textcolor{red}{JE DOIS COMPRENDRE CE QUE EST UN ORNSTEIN UHLENBECK PROCESS}
        }
\end{frame}

%------------------------------------------------

%------------------------------------------------
\section{Conclusion}

\begin{frame}{Conclusion}

    \note{The main roadblock when we try to apply SOCM to more challenging problems is that the variance of the
        factor alpha(v, Xv, B) explodes when f and/or g are large, or when the dimension d is high. The control L2 error
        for the SOCM and cross-entropy losses remains high and fluctuates heavily due to the large variance of alpha
        The large variance of alpha is due to the mismatch between the probability measures induced by the learned
        control and the optimal control. Similar problems are encountered in out-of-distribution generalization for
        reinforcement learning, and some approaches may be carried over from that area (Munos et al., 2016).}
\end{frame}

%! so many links to do with RL and Monte Carlo Markov Chains which representent discretized differential equations
%------------------------------------------------

% \begin{frame}{Blocks of Highlighted Text}
%     In this slide, some important text will be \alert{highlighted} because it's important. Please, don't abuse it.

%     \begin{block}{Block}
%         Sample text
%     \end{block}

%     \begin{alertblock}{Alertblock}
%         Sample text in red box
%     \end{alertblock}

%     \begin{examples}
%         Sample text in green box. The title of the block is ``Examples".
%     \end{examples}
% \end{frame}


\begin{frame}{References}
    \footnotesize
    \bibliography{reference.bib}
    \bibliographystyle{apalike}
\end{frame}

%------------------------------------------------

\end{document}